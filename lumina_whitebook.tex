
% Lumina_Whitebook_v1.0_ConservationAsIdentity.tex
% Camera-ready shell (LuaLaTeX/XeLaTeX). CC BY-SA 4.0
\documentclass[11pt,a4paper]{article}
\usepackage[margin=1in]{geometry}
\usepackage{microtype}
\usepackage{fontspec}
\usepackage{unicode-math}
\usepackage{hyperref}
\usepackage{bookmark}
\usepackage{graphicx}
\usepackage{tikz}
\usepackage{amsmath,amssymb,amsthm}
\usepackage{mathtools}
\usepackage{xcolor}
\usepackage{enumitem}

\IfFontExistsTF{Libertinus Serif}{\setmainfont{Libertinus Serif}}{\setmainfont{TeX Gyre Pagella}}
\IfFontExistsTF{Libertinus Math}{\setmathfont{Libertinus Math}}{\setmathfont{TeX Gyre Pagella Math}}

\definecolor{luminaGold}{HTML}{B7892B}

\hypersetup{
  pdftitle   = {Lumina Whitebook: Conservation as Identity},
  pdfauthor  = {Yay — The Lumina Project},
  pdfsubject = {Partition (Onu) formalism of conservation as identity},
  colorlinks = true, linkcolor=luminaGold, citecolor=luminaGold, urlcolor=luminaGold
}

\newcommand{\Partition}{\Pi}
\newcommand{\Forget}{\mathcal{F}}
\newcommand{\Hvis}{\mathcal{H}^{+}}
\newcommand{\Hled}{\mathcal{H}^{-}}

\title{\textbf{Lumina: Conservation as Identity}\\\large Nothing Vanishes; It Only Moves}
\author{Yay — The Lumina Project}
\date{Version v1.0 — CC BY-SA 4.0}

\begin{document}\maketitle

\begin{abstract}\noindent
We present \emph{Lumina}, a framework where conservation is elevated from law to identity. Every transformation partitions into two inseparable records: a visible flux and an invisible ledger. Classical subtraction is reinterpreted as the forgetful projection of a deeper partition operator. We formalize the partition morphism, connect it to relativistic pressure, and show classical physics as the ledgerless projection.
\end{abstract}

\section*{Axiom: No Physics Erased}
\begin{equation}
\boxed{\,\Box p \ \wedge\ \Box M(\partial_\mu J) \ =\ 0\,}
\end{equation}

\section*{Partition Operator}
For any extensive quantity \(X\),
\begin{equation}
\Partition(X)=(X_a,X_\ell), \qquad \frac{d}{dt}(X_a+X_\ell)=0.
\end{equation}
Here \(\Partition:\mathcal{E}\to\mathcal{E}\times\mathcal{L}\) and the forgetful functor \(\Forget:\mathcal{E}\times\mathcal{L}\to\mathcal{E}\). Classical subtraction corresponds to \(\Forget\circ\Partition\).

\begin{center}
\begin{tikzpicture}[>=latex,thick,node distance=2.8cm]
\node (E) {$\mathcal{E}$};
\node (EL) [right=of E] {$\mathcal{E}\times\mathcal{L}$};
\draw[->] (E)--node[above]{\small$\Partition$}(EL);
\draw[->,bend right=18] (EL) to node[below]{\small$\Forget$} (E);
\end{tikzpicture}
\end{center}

\section*{Relativistic Pressure Partition (one form)}
\begin{equation}
p = (\cos\chi + 1)\,y^{3} + E_{B} + E\,e^{B-C}.
\end{equation}

\section*{Unified Conservation Identity}
\begin{equation}
\mathbb{V} = p + \mathcal{C} + \mathbf{I}(\partial_t Dp) = 1.
\end{equation}

\section*{Continuum Limit}
\[
\text{Classical Physics} = \text{Ledgerless Projection of Lumina Conservation.}
\]

\bigskip
\noindent\textbf{License:} CC BY-SA 4.0. \quad \textbf{Contact:} yay@lumina.example
\end{document}
